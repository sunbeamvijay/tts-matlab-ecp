\documentclass[12pt,a4paper]{article}
\usepackage{latexsym}
\usepackage{amsmath}
\usepackage[dvips]{graphicx}
%\usepackage{times}

% Mathematical definition
\usepackage{amsthm}
\newtheorem{property}{Property}%[section]
\newtheorem{theorem}{Theorem}%[section]
\newtheorem{lemma}{Proposition}%[section]
\theoremstyle{definition}
\newtheorem{assumption}{Assumption}%[section]
\newtheorem{define}{Definition}%[section]
\newtheorem{definition}{Definition}%[section]

% Algorithm
\usepackage[ruled]{algorithm2e}

% Useful Mathematical command
\usepackage{amsfonts}% to get the \mathbb alphabet
\newcommand{\field}[1]{\mathbb{#1}}
\newcommand{\C}{\field{C}}
\newcommand{\N}{\field{N}}
\newcommand{\K}{\field{K}}
\newcommand{\R}{\field{R}}
%\newcommand{\M}{\mathcal{M}}
\providecommand{\abs}[1]{ \left| #1  \right|}%{\lvert#1\rvert}
\providecommand{\norm}[1]{\left\Vert #1 \right\Vert}
\newcommand{\mathand}{\quad\text{ and }\quad}


\begin{document}

\tableofcontents

\section{Theorie}
\subsection{Entropie (en th�orie de l'information)}

En th�orie de l'information, l'entropie de Shannon est la mesure de l'incertitude associ�e avec une variable al�atoire. Elle quantifie l'information contenu dans un message habituellement en bits ou en bits par symboles. Elle correspond � la longueur minimale du message necessaire pour communiquer l'information. C'est donc la limite absolue pour la meilleur m�thode de compression sans perte.

Pour une variable al�atoire discr�te $X$, qui peut prendre les valeurs $\left\{ x_1,\dots,x_n \right\}$, cette entropie est �gale �:
\[ H(X)  =  \operatorname{E}( I(X) ) = - \sum_{i=1}^np(x_i)\log_2 p(x_i) \]
%
o� $I(X)$ est la quantit� d'information de $X$ et $p(x_i)=Pr(X=x_i)$ est la probabilit� d'occurence de $x_i$.

Avant de voir les propri�t�s de $H$, il est important de remarqu� que ca definition d�pend du mod�le utilis� pour calculer la $p(x_i)$. En effet, un modele plus complexe (utilisant le contexte par exemple) permettra souvent de faire des predictions plus pr�cise, et l'entropie $H$ diminura. C'est pourquoi, dire que $H$ ``correspond � la longueur minimale du message necessaire pour communiquer l'information'' est assez relatif et depend du mod�le probabiliste.

%http://en.wikipedia.org/wiki/Information_entropy

L'entropie de Shannon est caract�ris�e par les propri�t�s suivantes.



\subsection{codage entropique}
=> huffman <=
arithmetic
\subsection{model}
=> static <=

/dynamic
avec contexe PPM

=> dictionnaire Lempel-Ziv-Welch (LZW) <=

DEFLATE is a lossless data compression algorithm that uses a combination of the LZ77 algorithm and Huffman coding => zipformat

\section{Fonctions Matlab}

sum
zeros

==, >, < vectorialise

plot, bar

sort

isstruct

\{ \} => cells

reshape

char

cree function et function intern

trouver lena plus petit

\section{TD}


\subsection{Compression sans perte}
Ecrire une fonction renvoyant la probabilite statique d'apparation d'un symbole dans un message. On prendra comme alphabet les nombres de 0 � 255, ce qui correspond � un octet.

Tracer le resultat pour l'image de lena et le texte

calculer l'entropie du message ordre 0, ordre 1

message: texte, image, xml/html\dots

donner la construction de l'arbre de huffman

% biblio
%http://en.wikipedia.org/wiki/Information_theory
%http://www.cs.cmu.edu/afs/cs/project/pscico-guyb/realworld/www/compression.pdf
%http://en.wikipedia.org/wiki/DEFLATE
%http://en.wikipedia.org/wiki/JPEG

\end{document}
