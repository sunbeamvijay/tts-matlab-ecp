\documentclass[12pt,a4paper]{article}
\usepackage{latexsym}
\usepackage{amsmath}
\usepackage[dvips]{graphicx}
%\usepackage{times}

% Mathematical definition
\usepackage{amsthm}
\newtheorem{property}{Property}%[section]
\newtheorem{theorem}{Theorem}%[section]
\newtheorem{lemma}{Proposition}%[section]
\theoremstyle{definition}
\newtheorem{assumption}{Assumption}%[section]
\newtheorem{define}{Definition}%[section]
\newtheorem{definition}{Definition}%[section]

% Algorithm
\usepackage[ruled]{algorithm2e}

% Useful Mathematical command
\usepackage{amsfonts}% to get the \mathbb alphabet
\newcommand{\field}[1]{\mathbb{#1}}
\newcommand{\C}{\field{C}}
\newcommand{\N}{\field{N}}
\newcommand{\K}{\field{K}}
\newcommand{\R}{\field{R}}
%\newcommand{\M}{\mathcal{M}}
\providecommand{\abs}[1]{ \left| #1  \right|}%{\lvert#1\rvert}
\providecommand{\norm}[1]{\left\Vert #1 \right\Vert}
\newcommand{\mathand}{\quad\text{ and }\quad}


\begin{document}

\tableofcontents

\section{Compression image avec perte}
\subsection{methods}
Reduction l'espace des couleurs par utilisation d'un palette/reduction du nombre de bit par pixel
=> some GIF/BMP, interessant citer DXT1

Sous echantillonage de la chrominance

Transform coding habituellement avec des operateurs de type fourier (DCT, wavelet)

fractal

\subsection{le format JPEG}
Color space transformation

Downsampling

Block splitting

Discrete cosine transform

Quantization

Entropy coding (voir section suivante)

\subsection{DCT}

\subsubsection{introduction}

\subsubsection{DCT et inverse DCT}

\subsubsection{DCT 2D}

% DXT1-5, other quantised

\section{Compression sans perte}
\subsection{theory}
\subsection{codage entropique}
=> huffman <=
arithmetic
\subsection{model}
=> static <=

/dynamic
avec contexe PPM

=> dictionnaire Lempel-Ziv-Welch (LZW) <=

DEFLATE is a lossless data compression algorithm that uses a combination of the LZ77 algorithm and Huffman coding => zipformat

\section{Fonctions Matlab}

sum
zeros

==, >, < vectorialise

plot, bar

sort

isstruct

\{ \} => cells

reshape

char

cree function et function intern

trouver lena plus petit

\section{TD}

PSNR + SSIM

Ajouter une partie theorie sur les mesure d'erreur de reconstitution de l'image

Fonctions fournies:
\begin{itemize}
  \item BlockSplitting - decoupe l'image en bloc de 8x8, resultat un vecteur de cell contenant des matrices 8x8
  \item DefaultQuantizationMatrix - renvoit la matrices de quantification standard
  \item NumberNZcoef - prend un vecteur de cell contenant des matrices 8x8 en entree et renvoit le nombre de valeurs non zeros
\end{itemize}

Ecrire une fonction dct88 et une fonction idct88.
Ecrire une fonction de quantification

Ecrire la fonction d'encodage, ecrire la fonction de decodage

Ecrire des function pour evaluer l'erreur de l'image reconstitue.

etudier influence taux de compression et matrice de quantification

bonus ecrire une fonction qui stream l'image compresse => appliquer compression sans perte


\subsection{Compression sans perte}
Ecrire une fonction renvoyant la probabilite statique d'apparation d'un symbole dans un message. On prendra comme alphabet les nombres de 0 � 255, ce qui correspond � un octet.

Tracer le resultat pour l'image de lena et le texte

calculer l'entropie du message ordre 0, ordre 1

message: texte, image, xml/html\dots

donner la construction de l'arbre de huffman

% biblio
%http://en.wikipedia.org/wiki/Information_theory
%http://www.cs.cmu.edu/afs/cs/project/pscico-guyb/realworld/www/compression.pdf
%http://en.wikipedia.org/wiki/DEFLATE
%http://en.wikipedia.org/wiki/JPEG

\end{document}
